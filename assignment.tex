\documentclass{article}
\usepackage[utf8]{inputenc}
\usepackage{amsmath}
\usepackage{amssymb}
\usepackage{graphicx}

\title{assignment}

\begin{document}

\maketitle

\section{11th question -c}
Q)In the given figure -1 ABCDE is a pentagon inscribed in a circle such that AC is a diameter and side BC//AE.If $\angle BAC $=50$\circ$, find giving reasons

1)$\angle ACB $

2)$\angle EDC $

3)$\angle BEC $

also prove that BE is diameter

\begin{figure}
    \centering
    \includegraphics[width=\linewidth]{assign question pic.png}
    \caption{figure -1}
    \label{fig:my_label}
\end{figure}\\

sol : $\angle ABC$ is 90$^\circ$ because angle subtended with diameter as a chord in the circle is right angle.

So $\angle ACB$ = 180$^\circ$-50$^\circ$-90$^\circ$=40$^\circ$.

       And condition given in question is AE//BC .From geometry we can say that if two lines are parallel(AE,BC) and there exists a transversal(BE) the alternative interior angles are equal.
       
       So $\angle EAC$=$\angle ACB$=40$^\circ$.
       
       By the observation $\angle EAC$+$\angle CAB$ is 90$^\circ$.
       
       Therefore BE must be a diameter of circle.
       
       And AEDC is a cyclic quadrilateral.So opposite angles inside quadrilateral
       add up to 180$^\circ$.
       
       Therefore $\angle EDC$=180$^\circ$ - $\angle EAC$=140$^\circ$.
       
       For finding$\angle BEC$  consider EDCB quadrilateral,even in this quadrilateral consider opposite angles $\angle EBC$ and $\angle EDC$ and again they add up to 180$^\circ$.
       
       
       So $\angle EBC$=180$^\circ$ - $\angle EDC$=40$^\circ$.
       
       
       Finally our required angle$\angle BEC$ is 180-$\angle EBC$-$\angle BCE$=50$^\circ$.
            
\end{document}
